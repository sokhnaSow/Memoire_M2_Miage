\documentclass[12pt]{report}
\usepackage[english]{babel}
\usepackage{natbib}
\usepackage{url}
\usepackage[utf8x]{inputenc}
\usepackage{amsmath}
\usepackage{graphicx}
\graphicspath{{images/}}
\usepackage{parskip}
\usepackage{fancyhdr}
\usepackage{vmargin}
\usepackage{colortbl}
\usepackage{hyperref}
\setmarginsrb{3 cm}{2.5 cm}{3 cm}{2.5 cm}{1 cm}{1.5 cm}{1 cm}{1.5 cm}

\author{SOW Sokhna Maimouna} % Author

\makeatletter
\let\theauthor\@author
\renewcommand{\thesection}{\@arabic\c@section}


\makeatother

\pagestyle{fancy}
\fancyhf{}
\lhead{\theauthor}
\rhead{\rightmark}
\lfoot{Universite Paris Nanterre}
%\rfoot{Cosmo Consult}
\cfoot{\thepage}
\renewcommand{\footrulewidth}{0.4pt}%trait horizontal pour le pied de page

\begin{document}

%%%%%%%%%%%%%%%%%%
%%% First page %%%
%%%%%%%%%%%%%%%%%%

\begin{titlepage}
\begin{center}

\includegraphics[width=0.6\textwidth]{fac}\\[1cm]

{\large Méthodes Informatiques Appliquées à la Gestion de l'Entreprise}\\[0.8cm]


{\large \textbf{Mémoire Master 2 Classique}}\\[0.5cm]

\vfill

% Title
\rule{\linewidth}{0.5mm} \\[0.4cm]
{ \huge \bfseries Comment intégrer les blockchains dans les transactions bancaires ?  \\[0.4cm] }
\rule{\linewidth}{0.5mm} \\[1.5cm]

\vfill

% Author and supervisor
\noindent
\begin{minipage}{0.4\textwidth}
  \begin{flushleft} \large
    \emph{Auteur :}\\
   Sokhna Maimouna \textsc{SOW}\\
  \end{flushleft}
\end{minipage}%
\begin{minipage}{0.4\textwidth}
  \begin{flushright} \large
    \emph{Encadrant :} \\
   Mme Marie-Pierre \textsc{Gervais}\\
  \end{flushright}
\end{minipage}

\vfill

% Bottom of the page
{\large Année scolaire \\ 2017 - 2018}

\end{center}
\end{titlepage}

%page de garde
\thispagestyle{empty}
\newpage
~

\newpage
\section{Remerciements}
\hspace{1cm} On remercie tt le monde.\\ 

\hspace{1cm} On continue de remercier.\\ 


%%%%%%%%%%%%%%%%%%%%%%%%%%%%%%%%%%%%%%%%%%%%%%%%%%%%%%%%%%%%%%%%%%%%%%%%%%%%%%%%%%%%%%%%%
\newpage
\renewcommand{\contentsname}{Table des matières}
\tableofcontents
\pagebreak

%%%%%%%%%%%%%%%%%%%%%%%%%%%%%%%%%%%%%%%%%%%%%%%%%%%%%%%%%%%%%%%%%%%%%%%%%%%%%%%%%%%%%%%%%

\section{Introduction}
\hspace{1cm} Notre monde évolue au rythme des innovations et des nouvelles technologies. Aujourd'hui, nous sommes témoins d'une nouvelle révolution mondiale avec une portée assez difficile à mesurer mais mettant en avance des possibilités d'applications infinies. Cette révolution est marquée par le phénomène "\textbf{blockchain}". Elle est née du croisement d'une technologie cryptographique de pointe basée sur les registres distribués et d'un contexte sociologique opportun. Le cas de la blockchain est juste \textit{phénoménale}. En effet la rapidité de son développement technologique coïncide avec un contexte sociologique favorable, ce qui augmente les chances de tirer profit des grandes innovations technologiques en les transformant en un vrai usage. \\

\hspace{1cm} \textit{Derrière ce concept de blockchain se cachent les cryptomonnaies qui se démarquent des} monnaies traditionnelles sur différents plans. Non seulement elles sont dématérialisées, anonymes, sécurisées mais ces devises peuvent s'échanger entre elles et permettent de faire des transactions bancaires. Et pendant ce temps, nous avons un système bancaire avec un réseau trop organisé, encadré et centralisé qui est bien construit et trop hiérarchisé avec des dirigeants qui contrôlent tout le système et ont toujours le dernier mot.\\

\hspace{1cm} Face à cette situation, il me semble opportun de s'interroger sur la performance et l'avenir des banques. Puisque c'est assez difficile de contourner les banques, cela me pousse à chercher comment procéder à mettre en place en place les blockchains afin de faciliter le monde monétaire de demain? Comment intégrer les blockchains dans les transactions bancaires ? Quels sont les acteurs à impliquer ? Tellement de questions qui me motivent à faire à faire une étude approfondie et dans la mesure du possible trouver une solution pour faire cohabiter les banques et les blockchains.\\

\hspace{1cm} A travers ces lignes je vais d'abord expliquer ce que c'est la blockchain en partant de sa création jusqu'à son processus final en détaillant son architecture et voir les différentes blockchains qui existent de nos jours. Ensuite j'étudierais l'impact des blockchains dans l'actualité économique sans oublier de mettre en évidence les faiblesses de cette nouvelle technologique et répondre aux questions posées précédemment...... QUOI RAJOUTER OU COMMENT REFORMULER


%%%%%%%%%%%%%% Glossaire %%%%%%%%%%%%%%
\newpage
\section{Glossaire}
\begin{tabular}{|c|c|}
\hline 
\rowcolor{green}KEY & Value \\

\hline
Subprimes & Prêts immobiliers accordées aux américains ne remplissant \\ & pas les conditions pour un prêt classique \\ 

\hline 
BTC & Symbole utilisé pour représenter l'unité de compte du bitcoin \\ 

\hline 
\end{tabular} 

\newpage
\section{Fonctionnement de la blockchain}
\hspace{1cm}  La blockchain fonctionne principalement avec une monnaie ou avec un token, c'est à dire un jeton, programmable.\\  A COMPLETER....

	\subsection{Historique}
\hspace{1cm} Tout a commencé par un groupe de 'geek' auto-organisé, sans hiérarchie et sans État qu'on appelle des \textbf{crypto-anarchistes}. Ces crypto-anarchistes savaient coder les données de façon à ce que personne d'autre que le destinataire ne puisse les déchiffrer. Ils se sont mis à "crypter des monnaies" d'où la naissance des\textbf{ monnaies virtuelles} appelées aussi \textbf{cryptomonnaie}. Si ces cryptomonnaies remontent aux années 70, ce phénomène devient célèbre qu'en début de 2009 avec la création de la blockchain et du bitcoin.\\

\hspace{1cm} En effet vers 2008, il y a eu une crise financière des subprimes. Les banques déclenchent alors la planche à billet afin de créer des euros et des dollars pour maintenir le système bancaire à flots. Et c'est vers cette même période précise qu'un certain \textbf{Satoshi Nakamoto}, un pseudo qui lui est attribué, publie un livre blanc de 10 pages, dans un forum de discussion, appelé \textbf{\textit{Bitcoin - A Per to Per Electonic Cash System }} qui va révolutionner le monde monétaire Par la suite, plusieurs personnes ont prétendu être Nakamoto mais nous n'avons jamais pu découvrir le vrai cerveau qui se cachait derrière. Il est resté actif jusqu'à mi 2010 pour ensuite disparaître des radars. C'est ainsi qu'on assiste à la naissance  de la première monnaie virtuelle décentralisée, qui fonctionne principalement avec la blockchain. \\ 

\hspace{1cm} De ce fait la première blockchain est apparue avec la monnaie numérique bitcoin. Aussitôt le marché virtuel se met à utiliser la technologie de la blockchain et plusieurs activités on tourné autour de cette technologie naissante. Des plate-formes, des sites Internet et même des applications on été mis en place pour la gestion des monnaies virtuelles et une utilisation permanente des blockchains. A COMPLETER!!!!!!!!!!!!!!!!!!!

	\subsection{Architecture de la blockchain}
\hspace{1cm} C'est quoi cette fameuse blockchain? Comment elle est 

\hspace{1cm} Une blockchain est une technologie de stockage et de transmission d’information sécurisée. Elle constitue une base de données sécurisée et distribuée qui contient toute l’historique de tous des les échanges entre ses utilisateurs depuis sa création. \\

Cette base est partagée par les différents utilisateurs sans intermédiaire, donnant la possibilité à chacun de vérifier la validité de la chaîne. Les transactions effectuées entre les utilisateurs du réseau sont regroupées par blocs. Chaque bloc est validé par les noeuds du réseau appelés les “mineurs”, selon des techniques qui dépendent du type de blockchain, qui si elle est publique, fonctionne obligatoirement avec une monnaie programmable. \\

Les transactions effectuées entre les utilisateurs du réseau sont regroupées par blocs. Chaque bloc est validé par les noeuds du réseau appelés les “mineurs”, selon des techniques qui dépendent du type de blockchain. Dans la blockchain du bitcoin cette technique est appelée le “Proof-of-Work”, preuve de travail, et consiste en la résolution de problèmes algorithmiques.\\


Ici je prendrais Bitcoin qui est un bon exemple de monnaie. Dans la blockchain du bitcoin cette technique est appelée le “Proof-of-Work”,c’est à dire preuve de travail, et consiste en la résolution de problèmes algorithmiques. Une fois le bloc validé, il est horodaté et ajouté à la chaîne de blocs. La transaction est alors visible pour le récepteur ainsi que l’ensemble du réseau. Le caractère décentralisé de cette nouvelle technologie, couplé avec sa sécurité et sa transparence, fait des prouesses aujourd’hui.\\

Par ailleurs il n’existe pas de banque centrale qui produit cette monnaie bitcoin. En effet, il s’agit d’ordinateurs distincts qui appartiennent au même réseau Bitcoin et qui sont rémunérés contre un service. C’est ce qu’on appelle le \textbf{minage de Bitcoin}. Il est sécurisé par le procédé cryptographique, la preuve de calcul. La difficulté pour quiconque de résoudre ces preuves de calcul assure la sécurité de toutes les transactions.\\

Toute blockchain publique fonctionne nécessairement avec une monnaie ou un token (jeton) programmable. Bitcoin est un exemple de monnaie programmable.

	\subsection{Les types de blockchain}
A REFLECHIR POUR GARDER OU NON!!!

\newpage
\section{Blockchain \& Économie}
\hspace{1cm} L'acceptation des cryptomonnaies comme mode de paiement n’est que la pointe visible de l’iceberg, parce que la technologie blockchain sous-jacente peut faire beaucoup plus que cela. Cela inclut des domaines tels que la comptabilité, les processus de gestion, la sécurité des données et la logistique. Ce qui peut pousser à un propriétaire d'entreprise à chercher sur comment la blockchain aura-t-elle un impact sur son entreprise ? C’est exactement ce que nous allons voir dans cet article.

- EXPLICATION SYSTEM ACTUEL\\
- SOLUTION\\
- LIMITE ET RISQUE\\

    \subsection{Le monde du E-commerce}
\textbf{Vente en ligne}, \textbf{Achat en ligne} \textbf{Paypal}\\
\hspace{1cm} JEXPLIQUE  LE E-COMMERCE\\

\hspace{1cm} La blockchain serait un grand atout pour les e-commerçants. En effet la mise en place de cette technologie dans ce secteur supprimerait tous les intermédiaires entre vendeurs et acheteurs. Les e-commerçants pourront proposer des transactions se passant d'intermédiaires, donc ne plus verser de commissions aux plate-formes, aux organismes bancaires. Prenons un exemple sur Shopify et Paypal, qui sont des sociétés de paiement en ligne très populaires de grandes plate-formes de commerce. Elles prennent une commission de 1,5\% à 6\%. Cette redevance est transmise au client, ce qui rend les achats en ligne beaucoup plus chers.\\

\hspace{1cm} D'après \textbf{Thomas France}, co-fondateur de la Maison du Bitcoin à Paris, ce système pourrait faire économiser 3\% à 4\% des chiffres d'affaires des e-commerçants qui réalisent beaucoup d'opérations depuis de l'étranger. Effectivement, les transactions internationales sont souvent réalisées par l'intermédiaire des plate-formes tierces, les obligeant à débourser beaucoup de frais. Alors que les paiements en bitcoins ne génèrent aucun frais de transactions pour le vendeur. Ceci est un grand avantage pour les e-commerçants en quête de rentabilité absolue.\\

\hspace{1cm} L'inexistence des frais de transaction est dû 



EXPLICATION DES COUTS 
Si le bitcoin est une monnaie électronique, il est avant tout une technologie. Un protocole dont le code source présente la principale caractéristique d'être ouvert à tous (open source). Il prend la forme d'un réseau de pair à pair (peer-to-peer), sans autorité centrale ou intermédiaire telle qu'une banque. Par conséquent, personne en particulier et tout le monde à la fois, contrôle et possède les bitcoins. Ainsi, lorsqu'une transaction en bitcoins est réalisée entre deux parties, la monnaie ne passe par aucun intermédiaire et est échangée, pour ainsi dire, de gré à gré, tout comme l'argent liquide. À ce titre, rien ne justifie des frais de transactions, et une fois le paiement effectué, il devient irréversible. De fait, l'e-commerçant est également protégé contre d'éventuels frais, liés à des annulations de paiement de la part de l'acheteur. En cas de mécontentement de ce dernier, le cybermarchand peut toujours le rembourser à posteriori, contre le retour du produit acheté. En outre, cela limite les risques de fraudes à la carte bancaire. Mais cela ne l'exempt pas d'être confronté à d'autres risques, qu'il convient de connaître avant de se lancer dans une procédure d'intégration de bitcoins dans ses moyens de paiements.

REQ, permet à l’utilisateur d’effectuer des transactions de manière transparente, sécurisée et à moindre coût, via la blockchain. Leur livre blanc estime actuellement les frais de 0,05 à 0,5pcent par transaction, les frais diminuent à mesure que le volume du réseau augmente puisqu’une partie des frais perçus sont brûlés et partiellement utilisés pour financer le réseau.\\

\textbf{Le risque de la volatilité}
Mettre en place le paiement par bitcoins n'est pas sans présenter certains risques. En effet, les bitcoins sont soumis à la loi de l'offre et de la demande ainsi, leur prix augmente lorsque la demande est forte et à l'inverse, il diminue lorsque la demande baisse. Problème, le cours du bitcoin a la fâcheuse réputation d'être imprévisible et très instable, en témoignent les fortes fluctuations de son prix ces derniers mois, sur des périodes très rapprochées. Pour un e-marchand, le problème peut alors se poser en ces termes : comment être certain de recevoir le montant du prix affiché sur un bien ou un service vendu, si le cours du bitcoin varie subitement entre le moment où la transaction est réglée par l'acheteur, et le moment où la somme est encaissée par l'e-commerçant ? La réponse est simple : confier à une société tierce la conversion du bitcoin en une devise "traditionnelle".


L’utilisation de Request Network pour gérer les paiements en ligne supprime l’intermédiaire du processus de paiement traditionnel.


En utilisant la technologie blockchain, leur plateforme peut perturber l’industrie pour les transactions financières en fournissant des normes de sécurité élevées, des coûts réduits pour l’exécution des transactions et une expérience utilisateur satisfaisante et facile. L’utilisation de PayPal entraîne des frais élevés par rapport à un grand nombre de cryptomonnaies. La récente scission avec l’un de leurs principaux partenaires, Ebay, souligne ces inconvénients. Ebay a attribué cette division aux coûts de transaction élevés pour les vendeurs et aux options de paiement limitées pour les acheteurs. Les cryptomonnaies ont la capacité technologique de fournir des solutions pour les deux et peuvent supprimer le besoin d’un tiers comme PayPal. Les cryptomonnaies peuvent potentiellement être meilleures que les services de portefeuille numérique actuels pour les raisons suivantes :

Transactions instantanées avec des frais peu élevés
N’importe qui peut y avoir accès
Pas besoin de fournir des informations personnelles et financièrement sensibles à un tiers (PayPal, banques …)
Les paiements ne sont qu’une des nombreuses choses dans lesquelles la blockchain peut perturber l’industrie du commerce électronique.


    \subsection{Paiement par carte bancaire physique}
\textbf{paiement sur place}, \textbf{paiement aux bornes}
\hspace{1cm}

    \subsection{Les transactions sur papier}
\textbf{Billet à ordre}, \textbf{chèques},\textbf{Lettre de change}, \textbf{Traite}

\hspace{1cm} Nous avons différents types de transactions basés principalement sur les documents,  que nous utilisons presque tous les jours. 
\begin{enumerate}
    \item Billet à ordre
    \item Chèque
    \item lettre de change
    \item Traite
\end{enumerate}

\hspace{1cm} 

Les « smart contracts » ou contrats intelligents sont la véritable force disruptive derrière la technologie blockchain. Les cryptomonnaies comme système de paiement sont déjà une grande perturbation dans l’industrie financière, mais il ne s’applique qu’à une seule industrie. Les contrats intelligents sont la véritable application de la technologie blockchain. Les contrats intelligents sont des accords numériques entre des parties qui s’exécutent automatiquement. Il consiste en quelques lignes de code qui spécifient les détails du contrat entre deux ou plusieurs parties. Ces détails peuvent être n’importe quoi, un montant à payer, le transfert de documents, l’envoi d’un produit ou le seuil de consommation d’électricité.

Pour le commerce électronique, ces contrats intelligents permettent des transactions directes entre les vendeurs et les acheteurs. Les contrats intelligents peuvent être programmés pour s’exécuter uniquement lorsque les obligations spécifiées ont été remplies. Par exemple, un acheteur peut envoyer le prix déterminé d’un produit en crypto-monnaie au contrat. Le vendeur envoie la preuve de propriété au contrat intelligent et lie le contrat intelligent à l’entreprise transportant le produit vendu. Une fois que le vendeur a rempli toutes ses obligations, le contrat intelligent enverra automatiquement les fonds au porte-monnaie du vendeur. Ce n’est que l’une des nombreuses applications des contrats intelligents. La même logique que celle décrite ci-dessus peut également s’appliquer à l’ensemble des chaînes d’approvisionnement, aux procédures de comptabilité organisationnelle, à la gouvernance, à la logistique et à bien d’autres processus organisationnels.

\textbf{smart contracts} A VOIR ICI


    \subsection{Les transferts d'argent}
\textbf{Mandat cash}, \textbf{Prélèvement}, \textbf{Virement}, \textbf{Western union}
\hspace{1cm}

    \subsection{Prêts bancaires}
\hspace{1cm}

    \subsection{Pré-autorisation}
\hspace{1cm}

    \subsection{Vente à distance}
\hspace{1cm}
    
    
Au niveau de l'international, le processus de la transaction change. Les cartes de paiement acceptées sont de type EMV (Europay Mastercard Visa) ou AmericanExpress.

Ces cartes utilisent les réseaux bancaires internationaux pour effectuer les traitements monétaires. Par exemple, si un porteur achète un produit aux Etats Unis, le traitement s'effectuera en empruntant le réseau Automated Clearing House avant de rejoindre la plate forme nationale Français qui est le Groupement des Systèmes Interbancaires de Télécompensation.

A chaque achat effectué en dehors de l'euros, une opération de change est effectuée. Cette opération est facturée et un pourcentage est versé de la transaction est reversé à la banque pour rembourser le coût de traitement.

Pour les transactions effectuées au sein de l'U.E., le projet SEPA va bientôt permettre d'améliorer l'organisation des systèmes de paiement européen.


\newpage		
\section{Conclusion}
\hspace{1cm} Depuis, de nombreuses institutions bancaires s’y intéressent et réalisent des transactions mais le débat reste entier : doit-on réguler ces monnaies virtuelles ? Comment contrôler la fluctuation de celles-ci ? Retour sur l’histoire des crypto-monnaies en 17 dates clés.\\

LA BLOCKCHAIN S’EST IMPOSÉE CETTE ANNÉE COMME UN SUJET INCONTOURNABLE ET EST D’AILLEURS ANNONCÉE COMME L’UNE DES TENDANCES QUI VA RÉVOLUTIONNER L’EXPÉRIENCE CLIENT D’ICI 2030\\

En raison de la sécurité qu’offrent le réseau et la cryptographie, la technologie blockchain fournit un système sécurisé grâce auquel les particuliers et les entreprises peuvent directement interagir entre eux sans avoir besoin d’un intermédiaire. Les seuls frais mineurs qui seront payés sont pour le réseau derrière la blockchain pour la validation des transactions et la sécurisation du réseau. L’acheteur et le vendeur ne paient aucuns frais à une compagnie de marché parce que, techniquement, il n’y en a pas. Les platesformes à travers lesquelles le commerce électronique sera réalisé sont des applications blockchain. Étant donné que les blockchains sont décentralisées, il n’y a pas de partie centrale, ni de société, qui définit les règles et décide de la manière dont les utilisateurs traiteront les uns avec les autres. Les utilisateurs, donc les particuliers et les entreprises, déterminent le développement et le fonctionnement de la plateforme. Les développeurs créent la blockchain et la mettent constamment à jour, mais ils ne peuvent la mettre à niveau qu’avec le consensus de la communauté.\\

Un défaut majeur dans la façon dont nous stockons actuellement les données est qu’elles sont stockées dans un endroit central et contrôlées par une partie centrale. Les données sont le nouveau pétrole et les cybercriminels sont désireux de voler ces énormes bases de données. La cybersécurité nécessite d’importants investissements en capital et des réglementations strictes, ce qui décourage les flux de revenus. Puisque les blockchains sont décentralisées, les données sont également décentralisées. Oui, les cybercriminels peuvent pirater des individus, mais ils ne voleront que les informations de l’individu qu’ils piratent. Il est pratiquement impossible de pirater une blockchain entier.

\newpage
\section{Webographie}

\newpage
\section{Annexes}

\end{document}